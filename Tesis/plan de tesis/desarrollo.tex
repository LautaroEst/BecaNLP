
\subsection{Teoría, enfoque y métodos a utilizar}



\subsection{Estudios conexos}

El alumno ya tiene aprobadas las siguientes materias relacionadas con los temas involucrados en la tesis:
\begin{itemize}
    \item Probabilidad y Estadística (81.04)
    \item Señales y Sistemas (86.05)
    \item Procesos Estocásticos (86.09)
    \item Teoría de Detección y Estimación (86.55)
    \item Procesamiento del Habla (86.53)
    \item Procesamiento de Señales I y II (86.51 y 86.52)
    \item Redes Neuronales (86.54)
    \item Teoría de la Información y Decodificación (86.11)
    \item Procesamiento de Imágenes (86.56)
\end{itemize}

Además, el alumno aprobó el curso "N1: Procesamiento del lenguaje natural con redes neuronales" dictado por la Escuela de Ciencias Informáticas (ECI) en 2019, y entre los años 2019 y 2020 se desarrolló como becario del Laboratorio de Procesamiento de Señales en Comunicaciones en el proyecto titulado "Aplicaciones de procesamiento de lenguaje natural (NLP) en el contexto de Internet de las Cosas (IoT)", como parte del Programa de Becas de la Sectretaría de Ciencia y Técnica de la UBA. 

\subsection{Alcance de la tesis}

