El procesamiento del lenguaje natural estudia los métodos computacionales con los cuales una computadora es capaz de realizar correctamente una tarea que implica trabajar con lenguaje humano en algún nivel. Existen actualmente muchas tareas que tienen gran interés comercial, algunas de las cuales pueden ser:
\begin{itemize}
    \item La identificación del género literario al que pertenece una síntesis de un libro.
    \item La predicción de una enferemedad dada una descripción de síntomas de un paciente.
    \item La detección de opiniones positivas o negativas en una calificación de un producto en una página web como Mercado Libre.
    \item La asistencia virtual al usuario de un sistema operativo como Apple o Android. 
\end{itemize} 

Si bien existe mucho esfuerzo en la actualidad por resolver este tipo de tareas de una manera satisfactoria, la mayor parte del trabajo se realiza en texto en idioma inglés, dando lugar a una falta de bases de datos en español (sobre todo, español latinoamericano) y de algoritmos con un buen desempeño en este idioma. Más aún, por más que la tarea se realice en forma satisfactoria, muchas veces se desea evaluar el nivel de comprensión de lenguaje que es capaz de desarrollar el algoritmo al momento de resolver el la tarea. Por ello, un problema adicional a considerar es el diseño del proceso de evaluación de estos algoritmos, lo cual generamente implica (entre otras cosas) crear bases de datos en español que permitan identificar de manera no sesgada el nivel de comprensión del algoritmo sobre el problema.

Motivados por esta situación, el objetivo de la tesis será el estudio de los algoritmos de procesamiento del lenguaje en español que requieren una comprensión profunda del mismo para resolver una determinada tarea. Para ellos será necesario crear bases de datos propias en este idioma y desarrollar un procedimiento de evaluación de dichos algoritmos para mostrar su nivel de comprensión en la tarea.