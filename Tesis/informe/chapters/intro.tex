\section{Procesamiento del Lenguaje Natural}

% En este capítulo hay que explicar qué es NLP y cómo lo vamos a usar nosotros. Es decir, si bien abarca una gran cantidad de temas y enfoques distintos, nosotros nos vamos a reducir a estudiar cómo hacer tareas que involucran lenguaje a partir de algoritmos de aprendizaje.

% Puntos a desarrollar:

% \begin{itemize}
% \item Introducción a la lingüística computacional, el procesamiento del lenguaje natural y esas cosas.
% \item Tareas de NLP importantes y cómo influye el hecho de conocer el lenguaje para resolverlas.
% \end{itemize}

\begin{itemize}
    \item Introducción general a NLP, lingüística computacional. Lenguaje en español. Estado del arte cualitativo.
    \item Introducción a las herramientas matemáticas para estudiar lenguaje. Aprendizaje estadístico del lenguaje. Decisión bayesiana y clasificación.
    \item Benchmarks para la evaluación de tareas. Explicar las diversas tareas y el hecho de que es importante desempeñarse bien en todas ellas. 
    \item Datos y sesgos. Hay que tener en cuenta que el procesamiento de lenguaje se hace a partir de los datos que nosotros vamos a conseguir. O sea, si bien el objetivo es entender el lenguaje y automatizar tareas que trabajan con lenguaje, nada de esto tiene sentido si no se tienen datos para trabajar.
    Por eso, hay que tener en cuenta todos los truquitos de la parte de organización de los datos, las inferencias que podemos extraer a paritr de ellos, cómo transmitir la información y esas cosas. Por ejemplo, si tengo que hacer un dataset en algún momento, tengo que poder diseñar preguntas interesantes que puedan ser contestadas a partir de este dataset.
    \item Los próximos capítulos explican las tareas más importantes relacionadas con el objetivo de la tesis (que además de hablar de NLP, hablamos de NLP en castellano).
\end{itemize}
