\documentclass[12pt]{article}

% Input encoding:
\usepackage[utf8]{inputenc}


% Idioma español:
\usepackage[spanish,es-tabla]{babel}
\usepackage[autostyle=false,style=american]{csquotes}
\MakeOuterQuote{"}


% Gráficos e imágenes:
\usepackage{graphicx}
\usepackage{caption}
\usepackage{subcaption}

% Layout de la página:
\usepackage[a4paper,width=150mm,top=25mm,bottom=25mm,bindingoffset=6mm]{geometry}
\usepackage[bookmarks]{hyperref}

% Matemática:
\usepackage{amssymb}


\title{\LARGE \textbf{Aprendizaje Supervisado}}
\author{Lautaro Estienne}
\date{Junio de 2020}

\begin{document}

\newcommand{\datasup}[2]{{\left\lbrace \left( {#1}^{(i)},{#2}^{(i)} \right) \right\rbrace}_{i=1}^N}
\newcommand{\xbf}{\mathbf{x}}
\newcommand{\Xbf}{\mathbf{X}}
\newcommand{\ybf}{\mathbf{y}}
\newcommand{\Ybf}{\mathbf{Y}}
\newcommand{\Rbb}{\mathbb{R}}
\newcommand{\Zbb}{\mathbb{Z}}
\newcommand{\Nbb}{\mathbb{N}}
\newcommand{\Scal}{\mathcal{S}}

\maketitle

En estas notas vamos a explicar las bases para aprender a diseñar algoritmos de aprendizaje automático que se entrenan en forma supervisada. Además, se explican algunos de los algoritmos que se utilizan comunmente en la práctica y que suelen dar buenos resultados, así como también los fundamentos teóricos por los cuales éstos funcionan. Se busca que, luego de haber entendido el tema, sea posible aplicar estos algoritmos y entender cómo adaptarlos al problema particular. 

\section{Introducción}
\section{Procesamiento del Lenguaje Natural}

% En este capítulo hay que explicar qué es NLP y cómo lo vamos a usar nosotros. Es decir, si bien abarca una gran cantidad de temas y enfoques distintos, nosotros nos vamos a reducir a estudiar cómo hacer tareas que involucran lenguaje a partir de algoritmos de aprendizaje.

% Puntos a desarrollar:

% \begin{itemize}
% \item Introducción a la lingüística computacional, el procesamiento del lenguaje natural y esas cosas.
% \item Tareas de NLP importantes y cómo influye el hecho de conocer el lenguaje para resolverlas.
% \end{itemize}

\begin{itemize}
    \item Introducción general a NLP, lingüística computacional. Lenguaje en español. Estado del arte cualitativo.
    \item Introducción a las herramientas matemáticas para estudiar lenguaje. Aprendizaje estadístico del lenguaje. Decisión bayesiana y clasificación.
    \item Benchmarks para la evaluación de tareas. Explicar las diversas tareas y el hecho de que es importante desempeñarse bien en todas ellas. 
    \item Datos y sesgos. Hay que tener en cuenta que el procesamiento de lenguaje se hace a partir de los datos que nosotros vamos a conseguir. O sea, si bien el objetivo es entender el lenguaje y automatizar tareas que trabajan con lenguaje, nada de esto tiene sentido si no se tienen datos para trabajar.
    Por eso, hay que tener en cuenta todos los truquitos de la parte de organización de los datos, las inferencias que podemos extraer a paritr de ellos, cómo transmitir la información y esas cosas. Por ejemplo, si tengo que hacer un dataset en algún momento, tengo que poder diseñar preguntas interesantes que puedan ser contestadas a partir de este dataset.
    \item Los próximos capítulos explican las tareas más importantes relacionadas con el objetivo de la tesis (que además de hablar de NLP, hablamos de NLP en castellano).
\end{itemize}


\section{Modelos}

\begin{itemize}
\item Modelos probabilísticos sin aproximaciones (filtro de wienner, etc.)
\item Modelos lineales generalizados
\item GDA / LDA
\item Naive Bayes
\item SVM
\item Redes neuronales
\end{itemize}

\section{Learning Theory}
Por más que se conozca perfectamente la forma del modelo, se disponga de muchos datos o tengamos todos los recursos computacionales a nuestra disposición, existen muchas dificultades para diseñar los algoritmos de aprendizaje automático que son intrínsecas al planteo del problema. Algunas decisiones como la cantidad de datos a recolectar para entrenar el modelo, la cantidad de parámetros de la arquitecturea o la tasa de aprendizaje, suelen ser algunas de las decisiones que más influencia tienen en el desempeño final del algoritmo. Por lo tanto, es importante estudiar la teoría que permite justificar cada una de estas decisiones.

\end{document}