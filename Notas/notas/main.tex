\documentclass[12pt]{article}

% Input encoding:
\usepackage[utf8]{inputenc}


% Idioma español:
\usepackage[spanish,es-tabla]{babel}
\usepackage[autostyle=false,style=american]{csquotes}
\MakeOuterQuote{"}


% Gráficos e imágenes:
\usepackage{graphicx}
\usepackage{caption}
\usepackage{subcaption}

% Layout de la página:
\usepackage[a4paper,width=150mm,top=25mm,bottom=25mm,bindingoffset=6mm]{geometry}
\usepackage[bookmarks]{hyperref}

% Matemática:
\usepackage{amssymb}


\title{\LARGE \textbf{Aprendizaje Supervisado}}
\author{Lautaro Estienne}
\date{Junio de 2020}

\begin{document}

\newcommand{\datasup}[2]{{\left\lbrace \left( {#1}^{(i)},{#2}^{(i)} \right) \right\rbrace}_{i=1}^N}
\newcommand{\xbf}{\mathbf{x}}
\newcommand{\Xbf}{\mathbf{X}}
\newcommand{\ybf}{\mathbf{y}}
\newcommand{\Ybf}{\mathbf{Y}}
\newcommand{\Rbb}{\mathbb{R}}
\newcommand{\Zbb}{\mathbb{Z}}
\newcommand{\Nbb}{\mathbb{N}}
\newcommand{\Scal}{\mathcal{S}}

\maketitle

En estas notas vamos a explicar las bases para aprender a diseñar algoritmos de aprendizaje automático que se entrenan en forma supervisada. Además, se explican algunos de los algoritmos que se utilizan comunmente en la práctica y que suelen dar buenos resultados, así como también los fundamentos teóricos por los cuales éstos funcionan. Se busca que, luego de haber entendido el tema, sea posible aplicar estos algoritmos y entender cómo adaptarlos al problema particular. 

\section{Introducción}
\section{Seccion 1}

Texto de la sección 1 del capítulo 1

\section{Modelos}

\begin{itemize}
\item Modelos probabilísticos sin aproximaciones (filtro de wienner, etc.)
\item Modelos lineales generalizados
\item GDA / LDA
\item Naive Bayes
\item SVM
\item Redes neuronales
\end{itemize}

\section{Learning Theory}
Por más que se conozca perfectamente la forma del modelo, se disponga de muchos datos o tengamos todos los recursos computacionales a nuestra disposición, existen muchas dificultades para diseñar los algoritmos de aprendizaje automático que son intrínsecas al planteo del problema. Algunas decisiones como la cantidad de datos a recolectar para entrenar el modelo, la cantidad de parámetros de la arquitecturea o la tasa de aprendizaje, suelen ser algunas de las decisiones que más influencia tienen en el desempeño final del algoritmo. Por lo tanto, es importante estudiar la teoría que permite justificar cada una de estas decisiones.

\end{document}