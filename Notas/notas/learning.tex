Por más que se conozca perfectamente la forma del modelo, se disponga de muchos datos o tengamos todos los recursos computacionales a nuestra disposición, existen muchas dificultades para diseñar los algoritmos de aprendizaje automático que son intrínsecas al planteo del problema. Algunas decisiones como la cantidad de datos a recolectar para entrenar el modelo, la cantidad de parámetros de la arquitecturea o la tasa de aprendizaje, suelen ser algunas de las decisiones que más influencia tienen en el desempeño final del algoritmo. Por lo tanto, es importante estudiar la teoría que permite justificar cada una de estas decisiones.