\documentclass[12pt]{article}

% Input encoding:
\usepackage[utf8]{inputenc}


% Idioma español:
\usepackage[spanish,es-tabla]{babel}
\usepackage[autostyle=false,style=american]{csquotes}
\MakeOuterQuote{"}


% Gráficos e imágenes:
\usepackage{graphicx}
\usepackage{caption}
\usepackage{subcaption}

% Layout de la página:
\usepackage[a4paper,width=150mm,top=25mm,bottom=25mm,bindingoffset=6mm]{geometry}

% Matemática:
\usepackage{amssymb}


\title{Learning Theory}
\author{Lautaro Estienne}
\date{Junio de 2020}

\begin{document}

\newcommand{\datasup}[2]{{\left\lbrace \left( {#1}^{(i)},{#2}^{(i)} \right) \right\rbrace}_{i=1}^N}
\newcommand{\xbf}{\mathbf{x}}
\newcommand{\Xbf}{\mathbf{X}}
\newcommand{\ybf}{\mathbf{y}}
\newcommand{\Ybf}{\mathbf{Y}}
\newcommand{\Rbb}{\mathbb{R}}
\newcommand{\Zbb}{\mathbb{Z}}
\newcommand{\Nbb}{\mathbb{N}}
\newcommand{\Scal}{\mathcal{S}}

\maketitle

El objetivo de estas notas es explicar la teoría necesaria para entender las bases de cómo diseñar modelos para clasificación y para regresión, dos problemas muy importantes en el ámbito de aprendizaje supervisado. 

\section{Aprendizaje Supervisado}
Acá comienza la sección 1.


\end{document}